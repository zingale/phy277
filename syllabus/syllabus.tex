%\DocumentMetadata{testphase={phase-III,math}}
\documentclass[10pt]{article}

\usepackage[tagged]{accessibility}


%\usepackage[cm]{fullpage}
\usepackage[lmargin=0.5in,rmargin=0.5in,
            tmargin=0.4in,bmargin=0.75in]{geometry}

\usepackage[parfill]{parskip}

\usepackage{hyperref}
%\usepackage{footnote}
\usepackage{palatino}

% URLs (special font for monospace)
%\usepackage{inconsolata}

\usepackage[defaultsans]{cantarell} % helvet} %
\usepackage[T1]{fontenc}

%\usepackage[small,compact]{titlesec}
%\titlespacing{\subsection}{0pt}{*1}{-0.125\parskip}
%\titleformat*{\subsection}{\sffamily\bfseries}

\usepackage{tcolorbox}

%\usepackage{fancyhdr}

%\pagestyle{fancy}


\usepackage{colortbl}

% multicolumn row coloring: http://www.latex-community.org/viewtopic.php?f=5&t=2565
\newcommand{\SetRowColor}[1]{\noalign{\gdef\RowColorName{#1}}\rowcolor{\RowColorName}}
\newcommand{\mymulticolumn}[3]{\multicolumn{#1}{>{\columncolor{\RowColorName}}#2}{#3}}
\definecolor{myGray}{rgb}{0.95,0.95,0.95}
\definecolor{examGray}{rgb}{0.90,0.90,0.90}
\definecolor{snowGray}{rgb}{0.95,0.95,0.95}

% color pairs of rows.  See http://tex.stackexchange.com/questions/59250/alternating-row-color-two-at-a-time-in-tables
%\usepackage[table]{xcolor}

%% \rowcolors{0}{myGray}{white}
%% \newcount\xrownum
%% \makeatletter
%%  \def\@rowc@lors{\noalign{%
%%   \global\advance\xrownum\@ne
%%   \ifodd\xrownum
%%   \global\advance\rownum\@ne
%%   \fi
%%   }\@rowcolors}



\newenvironment{itemsquish}
  { \begin{itemize}
    % set spacing between items
    \addtolength{\itemsep}{-0.25\baselineskip}
    % set spacing between lines
    \addtolength{\baselineskip}{-0.25\baselineskip} }
  { \end{itemize} }


% some commands to allow a large table to co-exist on a page with text
% see http://www.eng.cam.ac.uk/help/tpl/textprocessing/squeeze.html
% ``Squeezing Space in LaTeX''
\renewcommand\floatpagefraction{.9}
\renewcommand\topfraction{.9}
\renewcommand\bottomfraction{.9}
\renewcommand\textfraction{.1}
\setcounter{totalnumber}{50}
\setcounter{topnumber}{50}
\setcounter{bottomnumber}{50}

\hypersetup{
  pdftitle={PHY 277 Syllabus (Spring 2026)},
  pdflang={en-US}
}


\begin{document}

\begin{center}
{\large \sffamily \bfseries PHY 277: Computation for Physics and Astronomy / Spring 2026} {\footnotesize (rev.\ 1.01; 2026-01-01)}\\[1.5mm]
{\em Instructor:} Prof.\ Michael Zingale, ESS 452, michael.zingale@stonybrook.edu \\
{\em Class Meeting Time/\/Place:} Mon., Wed., Fri. 8:30~am to 9:20~am, location: Physics S-235 \\
\end{center}


\begin{tcolorbox}
\subsection*{Learning Outcomes}
An introduction to computing on UNIX/Linux computers. Fundamentals of
using UNIX/Linux to write computer programs for numerical algorithms
to solve computational physics and astronomy problems. Assignments are
carried out in a high-level compiled programming language such as
modern Fortran or C++ and require extensive use of SINC site computers
outside the classroom.
\end{tcolorbox}

\subsection*{Prerequisite}
PHY 125--127 or PHY 131--2 or PHY 141--2 (and the associated labs for
your sequence); AMS 151 or MAT 126 or MAT 131 or MAT 141


\subsection*{Course Website / Syllabus}

All course material will be available at:\newline
\url{https://zingale.github.io/phy277/}


\subsection*{Office Hours}

Office hours will follow immediately after class on Monday and
Wednesday.  It is not possible to pick office hours that can
accommodate the schedule of all students in this class.  You are
encouraged to contact the instructor to make an appointment outside of
these times.


\subsection*{Homework}

There will be 8--10
homework assignments throughout the course.  Students will typically
have 1 week to complete an assignment.  While it is recognized that
students sometimes work together and discuss the homeworks as part of
the learning process, {\em what you turn in must be your own work.}
{\em Copying will not be tolerated.}

For assignments where you are writing code, you must provide the
machine-readable source code (uploaded to Brightspace).  The code
must compile without error on the class machines (GCC 15).  Points
will be deducted if the code does not compile.

Homeworks are due at the time/date listed on the assignment.  {\bf No
  late homeworks will be accepted}.  Homework grades will be posted to
the Brightspace gradebook approximately 1 week after the due date,
with grading comments available in the Brightspace gradebook.
Students should report any errors/missing grades promptly.

{\bf AI/LLM/ChatGPT policy}: you may {\em not} use ChatGPT or similar
AI / large-language models for the homeworks.  This will be treated
as an academic integrity violation.





\subsection*{Exams}

There will be 2 midterms and one final exam.  For each of the exams,
students are responsible for knowing the material presented in the
lectures, recitations, assigned as homework, and in the assigned
chapters of the text.  Exams will take place in-class---{\bf we will
  not use the common exam period scheduled by the registrar for the
  midterms}.

{\em Students should not expect that they will be allowed to make up
  an exam without advanced notice.}  Reasons for wanting to make-up an
exam will be judged on a case-by-case basis.  Students wanting to make
up an exam must have a {\em valid} excuse (e.g.\ athlete in
University-related sporting event, jury duty, medical emergency) and
notify the instructor {\em before} the scheduled exam (or as soon as
is safely possible afterwards).  {\em No make-ups will be allowed more
  than one week after the original exam date.}  


\subsection*{Final Exam}

The final exam will be given at the time and date scheduled by the
registrar.



\subsection*{Course Topics}

This is a preliminary list of topics we will cover in the class.
Topics and dates will shift as the semester evolves.  The course
website will always be the authoritative source for what we cover.

\begin{itemize}
  \item Intro / logistics (week 1)
  \item Unix shell (weeks 1--3)
  \item C++ data types (weeks 4--5)
  \item C++ functions (weeks 6--7)
  \item Software Engineering (week 8)
  \item Classes (week 9--10)
  \item The C++ standard library (week 11--12)
  \item Numerical algorithms (week 13)
  \item Crash-course on python (if time) (week 14)
\end{itemize}

\subsection*{Course Grade}

The final grade will be based on the homeworks, midterm, and final
exam using the following weighting:
\begin{itemsquish}
\item homework: 20\%
\item class participation: 5\%
\item midterm 1: 25\%
\item midterm 2: 25\%
\item final exam: 25\%
\end{itemsquish}

Computed this way, the overall course grade will range from 0--100.
Letter grades will be based on a standard grade scale (i.e.\
an overall score $> 90/100$ would be an A- or better).  However, if
necessary, a curve will be applied to the overall course grade,
considering the overall performance of the class.
Students who wish to discuss their grades or class performance should
see the instructor in person.  {\em For privacy reasons, grades will
  not be discussed via e-mail.}

\subsection*{Student Accessibility Support Center Statement}

If you have a physical, psychological, medical, or learning disability
that may impact your course work, please contact the Student
Accessibility Support Center, Stony Brook Union Suite 107, (631)
632-6748, or at \href{mailto:sasc@stonybrook.edu}{sasc@stonybrook.edu}. They will determine with you what
accommodations are necessary and appropriate. All information and
documentation is confidential.

Students who require assistance during emergency evacuation are
encouraged to discuss their needs with their professors and Student
Accessibility Support Center. For procedures and information go to the
following website:
\url{https://ehs.stonybrook.edu/programs/fire-safety/emergency-evacuation/evacuation-guide-disabilities}
and search {\em Fire Safety and Evacuation and Disabilities}.

\subsection*{Academic Integrity}

Each student must pursue his or her academic goals honestly
and be personally accountable for all submitted work. Representing
another person's work as your own is always wrong. Faculty are
required to report any suspected instances of academic dishonesty to
the Academic Judiciary. Faculty in the Health Sciences Center (School
of Health Technology \& Management, Nursing, Social Welfare, Dental
Medicine) and School of Medicine are required to follow their
school-specific procedures. For more comprehensive information on
academic integrity, including categories of academic dishonesty,
please refer to the academic judiciary website at
\url{http://www.stonybrook.edu/commcms/academic\_integrity/}

\subsection*{Critical Incident Management}

Stony Brook University expects students to respect the rights,
privileges, and property of other people. Faculty are required to
report to the Office of Judicial Affairs any disruptive behavior that
interrupts their ability to teach, compromises the safety of the
learning environment, or inhibits students' ability to learn.  Faculty
in the HSC Schools and the School of Medicine are required to follow
their school-specific procedures.  Further information about most
academic matters can be found in the Undergraduate Bulletin, the
Undergraduate Class Schedule, and the Faculty-Employee Handbook.


\subsection*{Electronic Communication}

Email to your University email account is an important way
of communicating with you for this course.  For most students the
email address is `{\tt firstname.lastname@stonybrook.edu}'.
%, and the account can be accessed here.
{\em It is your responsibility to read your email received at this
  account.}  For instructions about how to verify your University
email address see this: \\[0.25em]
{\small \url{http://it.stonybrook.edu/help/kb/checking-or-changing-your-mail-forwarding-address-in-the-epo}}\\
If you choose to forward your University email to another account, we
are not responsible for undeliverable messages.

\subsection*{Religious Observances}

See the policy statement regarding religious holidays at\hfill\\
{\small \url{http://www.stonybrook.edu/commcms/provost/faculty/handbook/employment/religious_holidays_policy.php}} \linebreak
%

Students are expected to notify the course professors by email of
their intention to take time out for religious observance.  This
should be done as soon as possible but definitely before the end of
the `add/drop' period.  At that time they can discuss with the
instructor(s) how they will be able to make up the work covered.


\end{document}
